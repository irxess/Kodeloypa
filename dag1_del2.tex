\documentclass[a4paper]{article}
% needs to be compiled with xelatex
\usepackage{eso-pic,graphicx}
\usepackage{lipsum}
\usepackage{background}
\usepackage{fontspec}
%\usepackage[top=2cm, bottom=2cm, outer=0cm, inner=0cm]{geometry}
\usepackage[pass,a4paper]{geometry}

\setmainfont{Didact Gothic}
\linespread{1.4} % 1.6 is double
\backgroundsetup{
scale=1,
angle=0,
opacity=1,  %% adjust
contents={\includegraphics[width=\paperwidth,height=\paperheight]{background}}
}
\setlength{\parskip}{13pt}
\setlength\parindent{0pt}

\newcommand{\block}[2][-0.4]{\raisebox{#1\height}{\includegraphics{#2}}}

\begin{document}
\begin{large}

Det er litt kjedelig med bare en katt. Katten er en figur, også kalt «sprite» 
i Scratch. For å legge til nye figurer, klikk på \block{new_sprite}. Finn frem til 
figurene dere har valgt å bruke, og legg dem til i Scratch. Høyreklikk på katten
og trykk «delete» for å fjerne den.


Hver figur har sitt eget kodeområde. For å bytte mellom dem, dobbeltklikk på 
figuren din i området med oversikten over alle figurene. Når en figur er valgt, 
er det mulig å gi den et navn øverst til venstre. Gjør dette med alle figurene
dere har, slik at det er lettere å holde oversikt.

Nå som vi har flere figurer, er det mulig å få dem til å reagere på hverandre. 
I Scratch fungerer dette ved å la figurene sende meldinger til hverandre.
Dra ut \block{broadcast} (under «control»). Trykk på pilen for å lage en ny 
melding og gi den et fint navn.

Dobbeltklikk på en annen figur for å komme inn i kodeområdet til den. For å reagere
på meldingen, dra over \block{receive} . Husk å bytte til riktig melding hvis du har laget flere.

Oppgave
\begin{itemize}
    \item En figur sier noe og sender en melding til en annen figur. Den andre figuren venter et sekund og svarer.
    \begin{itemize}
        \item Hint: det er forskjell mellom å si noe og å sende en melding. Dere trenger to blokker for å gjøre begge deler.
    \end{itemize}
\end{itemize}

\newpage

Lyd er knyttet til en figur. Ikke alle figurer følger med lyd, så vi kan legge til egne lyder. 
For å legge til lyd til figuren som er markert, bytt til \block{sounds} og klikk «import».
Velg så lyden dere har lyst til å bruke. Lyden spilles av ved bruk av \block{play_sound}. 
Denne blokken ligger under \block{sound}.

De fleste figurer kan også endre utseende. For å se hvilke utseender figurene deres har, 
se under \block{costumes}. Å bytte mellom utseender gjøres med \block{switch_costume}.

Endring av den hvite bakgrunnen gjøres på samme måte. Blokkene har andre navn, 
men bakgrunnen fungerer på samme måte som en figur. Den har sitt eget kodeområde og kan endre utseende. 
Bakgrunnen har ikke blokker som får den til å flytte på seg.

Oppgaver
En av figurene bytter til et annet utseende når en annen figur lager lyd.
Bytt fargen til bakgrunnen hver gang man trykker mellomrom.

\begin{itemize}
    \item En av figurene bytter til et annet utseende når en annen figur lager lyd.
    \item Bytt fargen til bakgrunnen hver gang man trykker mellomrom.
\end{itemize}

\end{large}
\end{document}
